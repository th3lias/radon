\documentclass[a4paper,12pt]{amsart}
\usepackage{mathtools}
% Sets
\let\oldS\S
\newcommand{\R}{\mathbb{R}}
\renewcommand{\S}{\mathbb{S}}

% Relations
\DeclareMathOperator{\defn}{\, \stackrel{def}{=} \,}

% Differential Calculus
\let\oldd\d
\renewcommand{\d}{\mathrm{d}}
\newcommand{\intt}[4]{\int_{#1}^{#2} #3\, \d #4}

% Binary
\newcommand{\inner}[2]{\left\langle #1\, \mid\, #2 \right\rangle}

% Project specific
\newcommand{\cR}{\mathcal{R}}
\DeclareMathOperator{\rad}{R}
\begin{document}

\title{The Radon Transform}
\author{Elias Mindlberger}

\keywords{Radon Transform, Wavelets}

\date{\today}

\begin{abstract}
    This is an overview of a project for the subject ''Wavelets - Functional Analytical Basics'' in the summer semester of 2025 @ JKU Linz.
\end{abstract}

\maketitle


When a beam of X-Ray photons travels through a homogeneous material, the intensity of photons at the time of entering the material is high compared to the intensity at the moment of exit. This is described by the equation
\[
    I = I_0 e^{-\mu w}
\]
where \(I_0\) is the input intensity and 
\(I\) is the observed intensity after the beam passes the 
distance \(w\) through the material. The \emph{attenuation} coefficent \(\mu\) 
depends (among other things) on the density of the material.
When the material is inhomogeneous, i.e. \(\mu\) represents a varying density, the 
corresponding equation for the observed intensity becomes \[
    I = I_0 \exp \left( - \intt{L}{}{\mu(x,y)}{s} \right)
\]
where \(L\) is the beam path parametrised by \(s\). By moving the source of the photons around the material, one obtains a set of line integrals over the density \(\mu\) and by taking logarithms, one obtains
samples of Radon transforms where the \(d\)--dimensional Radon transform of an integrable function \(f: \R^d \to \R\) is defined as
\[
    \rad f(\sigma, \theta) \defn \intt{\inner{x}{\theta} = \sigma}{}{f(x)}{x},\qquad (\sigma, \theta) \in \R \times \S^{d-1}.
\]

\end{document}